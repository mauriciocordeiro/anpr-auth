\documentclass[12pt]{article}

\usepackage{anpr}
\usepackage{graphicx,url}
\usepackage[utf8]{inputenc}
\usepackage[brazil]{babel} 
     
\sloppy

\title{Aplicação de \textit{Automatic Number Plate Recognition} (ANPR) no controle de acesso de veículos}

\author{Maurício de A. Cordeiro\inst{1} }


\address{Instituto Federal de Educação, Ciência e Tecnologia da Bahia\\ 
	Avenida Sérgio Vieira Melo, 3150. Bairro Zabelê - Vitória da Conquista - BA - Brasil\\
	CEP 45078-900
  \email{mauriciocordeiro@live.com}
}

\begin{document} 

\maketitle

\begin{abstract}
  \textit{soon...}
\end{abstract}
     
\begin{resumo} 
 \hbox{Este trabalho consiste no desenvolvimento de sistema para controle de acesso de veículos baseado em ANPR (\textit{Automatic Number Plate Recognition}), com largo potencial de sua aplicação em locais ou espaços físicos que exigem algum nível de segurança, quando da entrada e saída de pessoas (a título de exemplo, citam-se aqui condomínios e estacionamentos privados). }
\end{resumo}

\section{Introdução}

\subsection{Trabalhos correlatos}


\section{Arquitetura}

\cite{viggiato2018}

\subsection{Microsserviços}

\subsection{Conteiner}

\subsection{ANPR}

\section{"O trabalho"}


\section{Resultados}


\section{Conclusão}

\textit{trabalhos futuros}


\section{Referencias}


\bibliographystyle{anpr}
\bibliography{anpr}

\end{document}
